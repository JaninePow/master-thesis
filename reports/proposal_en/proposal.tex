\documentclass[12pt,a4paper]{article}
\usepackage{german}
\usepackage{times}
\usepackage{hyperref}
\usepackage{xspace}
\usepackage{microtype}
%\usepackage{doublespace}
%------------------------------------------------------------------------------
%\setstretch{1.0}
\voffset-5mm
\hoffset-5mm
\textwidth17cm
\textheight24cm
\headsep0mm
\headheight0mm
\oddsidemargin0.3mm
\pagestyle{empty}
\parindent0mm
\parskip1ex
%------------------------------------------------------------------------------
%==============================================================================

\providecommand{\etal}[1]{#1\emph{~et~al.\xspace}}
\renewcommand\refname{References}

\begin{document}



\begin{center}
	Master's Thesis at the Pattern Recognition Lab, FAU Erlangen-Nuremberg \hfill \\[5mm]
																				
	\mbox{}\\
	{\Large Determining the Influence of Papyrus Characteristics and Data Argumentation on Fragments Retrieval with Deep Metric Learning}
			
\end{center}
%Body


State-of-the-art end-to-end systems for ancient papyrus fragments retrieval use self-supervised deep metric learning (DML) \cite{Pirrone21}. More precisely, deep learning is used to pre-sort ancient papyri fragments that need to be reconstructed prior to their analysis by papyrologists. 
\\
\\
Deep learning has revolutionized the performance of papyrus fragments retrieval, but the approach is limited because it demands sufficient labeled data for training. In the low-data regime, parameters are under-determined, and learned networks generalize poorly \cite{Antoniou18}. Data Augmentation mitigates this by using existing data more effectively. However, standard data augmentation produces only limited plausible alternative data. Given the potential to generate a much broader set of augmentations, we train a generative adversarial-based model to do data augmentation \cite{Tensmeyer20, Antoniou18}. 
\\
\\
Furthermore, the characteristics of papyrus fragments have not been dealt with in-depth \cite{Tensmeyer20}. That means research has tended to focus on fragments retrieval with all characteristics rather than a subset of characteristics. With this in mind, we try to analyze the influence of foreground (text) and background characteristics (papyrus-fibers) on fragments retrieval.
\\
\\
The thesis consists of the following milestones: 

\begin{itemize}
	\item Incorporating thresholding \cite{Tensmeyer20, Pratikakis19} and impainting \cite{Liu18Impainting} into the training pipeline of a DML model \cite{Musgrave20}.
	      	      	      	      	      	      	      		      		      	      	      	      	      	      	      	      	      	      	
	\item Use the preprocessed data to create a large dataset suitable for DML network throughout a GAN-based method \cite{Antoniou18}.
	      	      	      	      	      	      	      		      		      	      	      	      	      	      	      	      	      	      
	\item Evaluate the performance of the thresholding and impainting techniques on the Michigan dataset \cite{Pirrone21} and the large dataset created by the GAN.
	      	      	      	      	      	      	      		      		      	      	      	      	      	      	      	      	      	      	
	\item Further experiments regarding fragment matching via papyri-fibers.
	      	      	      	      	      	      	      		      		      	      	      	      	      	      	      	      	      	      
\end{itemize}
		
		
The implementation will be done in Python.\\
		
\begin{tabular}{ll}
	\emph{Supervisors:} & Dr.-Ing.~V.~Christlein,  Prof.~Dr.-Ing.~habil.~A.~Maier, Mathias Seuret M. Sc.
	\\
	\emph{Student:}     & Timo Bohnstedt
	\\
	\emph{Start:}       & November 8th, 2021                                            \\
	\emph{End:}         & April oth, 2022                                        \\
\end{tabular}
\nopagebreak[4]
\small
\bibliographystyle{IEEEtran}       %TODO change bibliographystyle
\bibliography{proposal}
		
\end{document}
%==============================================================================
