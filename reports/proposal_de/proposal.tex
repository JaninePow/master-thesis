\documentclass[12pt,a4paper]{article}
\usepackage{german}
\usepackage{times}
\usepackage{hyperref}
\usepackage{xspace}
\usepackage{microtype}
%\usepackage{doublespace}
%------------------------------------------------------------------------------
%\setstretch{1.0}
\voffset-5mm
\hoffset-5mm
\textwidth17cm
\textheight24cm
\headsep0mm
\headheight0mm
\oddsidemargin0.3mm
\pagestyle{empty}
\parindent0mm
\parskip1ex
%------------------------------------------------------------------------------
%==============================================================================

\providecommand{\etal}[1]{#1\emph{~et~al.\xspace}}
\renewcommand\refname{References}

%inset reserach questions
\usepackage{enumitem}

\newlist{questions}{enumerate}{2}
\setlist[questions,1]{label=\bf{RQ\Roman*:},ref=RQ\Roman*}
\setlist[questions,2]{label=(\alph*),ref=\thequestionsi(\alph*)}


\begin{document}



\begin{center}
	Master's Thesis at the Pattern Recognition Lab, FAU Erlangen-Nuremberg \hfill \\[5mm]
																				
	\mbox{}\\
	{\Large Determining the Influence of Papyrus Characteristics and Data Argumentation on Fragments Retrieval with Deep Metric Learning}
			
\end{center}
%Body


In der Praxis sind altertümliche Papyri in mehre Fragmente zerrissen. 
Die Aufgabe von Papyrologen ist es, diese Fragmente zusammenzusetzen und zu entziffern. Zusammen mit Historikern können dadurch wichtige Ergebnisse über das Altertum gewonnen werden. Allerdings ist der Prozess des händischen Zusammensetzens sehr zeitaufwendig und schwierig. 
\\\\
In der folgenden Arbeit geht es darum, Papyrologen die Arbeit zu erleichtern, indem man den Prozess des Zusammensetzens teilautomatisiert. Hierfür wird ein Algorithmus entworfen, der zu einem Papyrus-Fragment eine kleinere Auswahl an möglicherweise passenden Fragmenten vorschlägt. Zusammengefasst wird dieser Algorithmus im folgenden eine Puzzlehilfe Funktion genannt. Mit solch einer Puzzlehilfe Funktion kann die Effektivität und Effizienz beim Zusammensetzten von Fragmenten durch Papyrologen erhöht werden. Allerdings muss eine Puzzlehilfe Funktion dabei mehre Bedingungen erfüllen.
\\\\
Zum einen darf das Erstellen einer solchen Funktion keine großen Datenmengen an zusammengesetzten Payri Fragmenten voraussetzen, denn es existieren weltweit nur wenige solcher Daten. Deep Metric Lerarning (DML) wurde bisher verwendet, um eine Puzzlehilfe Funktion zu erstellen und erzieht dabei eine gute Genauigkeit. Wenn man das DML weiter verfeinern möchte, um eine sehr gute Genauigkeit zu erreichen, werden immens viele Daten benötigt. Andernfalls würde das Model sehr wahrscheinlich nicht konvergieren. Wenn man die Fragmente händisch zusammensetzen würde, um somit einen großen Datensatz zu erhalten, wäre der Zweck der Puzzlehilfe Funktion obsolet. Der Zeitaufwand für den Papyrologen bleibt im besten Fall derselbe oder wird im schlechtesten Fall erhöht. Generative-Advesarial-Networks (GANs) sind eine gut erforschte Möglichkeit, domänenübergreifend mehr künstliche Daten zu erzeugen. Es wurde gezeigt, dass die Genauigkeit über verschiedene Anwendungsbereiche hinweg dadurch signifikant erhöht werden kann. Unser Ziel ist es diese Methode zu verwenden, um die Hürde von zu wenig Daten zu überwinden. 
\\\\
Eine andere Bedingung ist, dass die falsch-negativen Treffer möglichst gering sind. In der Vorauswahl der Puzzlehilfe Funktion sollen also möglichst alle potenziell passenden Fragmente enthalten sein. Ist das passende Fragment nicht in der Teilmenge enthalten, hat der Papyrologe keine Chance, das richtige Fragment zu identifizieren. Es sei denn er weitet seine Suche wieder auf den gesamten Datensatz aus, was die Puzzlehilfe Funktion wieder obsolet werden lässt. Eine weiter Nachteil ist, dass er möglicherweise extrem lange nach dem passenden Fragment sucht, welches aber nicht zu finden ist. 
Zielführende Metriken, um diese Eigenschaft zu Messen und ein Gesamtbild über das Model zu gewinnen, sind die mAP, top-1, pr@10, pr@100. Im Folgenden sprechen wir vereinfacht nur über Genauigkeit. Diese Genauigkeit hängt stark davon ab, welcher Bildausschnitt als Eingabe verwendet wird. Genauer, was dem Model beim Trainieren als gleich und ungleich suggeriert wird, determiniert die Genauigkeit des Modells. 
Deshalb soll unser Ansatz statistisch untersuchen, wie bestimmte Papyruscharakteristiken die Genauigkeit eines solchen Models beeinflussen. Hierfür vergleichen wir die Genauigkeit verschiedner Modelle, indem nur der Vordergrund eines Papyrus (Text) oder nur der Hintergrund (Fasern) als Eingabe zum trainieren verwendet wird. Das separieren von Text und Phasern ist eine ganz eigene Herausforderung mit vielen Hindernissen. Wir verwenden verschiedene Methoden und evaluieren diese an der Gesamtgenauigkeit. Eine separierte Evaluierung wäre ohne eine Pixel genaue Ground Truth nicht möglich. Außerdem sind wir am Einfluss an der Genauigkeit und nicht am Separieren selbst interessiert. Für die Texterkennung verwenden wir dabei erst einen Threshold, welcher den Text herauslöst und uns eine Maske zurückgibt. Die herausgelösten Stellen können nun mittels inpainting und der Maske aufgefüllt werden. Auch das Inpainting Verfahren wird nur anhand der Gesamtgenauigkeit evaluiert.
\\\\
Zuletzt untersuchen wir, ob bestimmte Charakteristiken (z.B. Fasern) noch auf andere weiße verwendet  werden können. Zum Beispiel, um den Papyrologen zu helfen, nicht nur potenzielle Kandidaten zu identifizieren, sondern auch deren genaue Position zu bestimmten. Diese wäre der logische Schritt hin zu einer vollautomatisierten Funktion. Also in unserer Terminologie eine Transformation von einer Puzzlehilfe zu einem Puzzlelöser.   


Zusammenfassend gliedert sich die Arbeit in folgende Meilensteine:


\begin{enumerate}
	\item Separieren von Text und Papyrus-Fasern durch Binarization und Inpainting.
	
	\item Erzeugen von größeren Datenmengen durch ein GAN.
	
	\item Evaluierung mittels eines DML Models anhand der originalen Daten im originalen Zustand (Text und Fasern), im bearbeiteten Zustand (nur Text oder nur Fasern). Außerdem eine Evaluierung der künstlichen Daten zusammen mit den originalen Daten im originalen Zustand und bearbeiten Zustand.
	
	\item Review des State-of the Art über die Möglichkeiten, bestimmte Charakteristiken für die Positionsbestimmung zu verwenden. 
	    	      	      	      	      	      	      
\end{enumerate}
		
Aus den Meilensteinen ergeben sich die folgenden Forschungsfragen:

\begin{questions}
	\item Unterscheiden sich die gewählten Metriken (mAP, top-1, pr@10, pr@100) signifikant anhand der verwendeten Binarization und impainting Methoden?
	
	\item  Unterscheiden sich die gewählten Metriken (mAP, top-1, pr@10, pr@100) signifikant wenn nur der Text oder nur die Fasern als Eingabe verwendet werden im Gegensatz zu den unbearbeiteten Daten?  
	\item  Unterscheiden sich die gewählten Metriken (mAP, top-1, pr@10, pr@100) signifikant, wenn zum trainieren zusätzlich künstliche Daten verwendet werden, welche durch ein GAN erzeugt wurden?
\end{questions}

		

		
\begin{tabular}{ll}
	\emph{Supervisors:} & Dr.-Ing.~V.~Christlein,  Prof.~Dr.-Ing.~habil.~A.~Maier, Mathias Seuret M. Sc.
	\\
	\emph{Student:}     & Timo Bohnstedt
	\\
	\emph{Start:}       & November 8th, 2021                                            \\
	\emph{End:}         & April oth, 2022                                        \\
\end{tabular}
\nopagebreak[4]
\small
\bibliographystyle{IEEEtran}       %TODO change bibliographystyle
\bibliography{proposal}
		
\end{document}
%==============================================================================
